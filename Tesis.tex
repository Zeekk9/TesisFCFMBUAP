%%%%%%%%%%%%%%%%%%%%%%%%%%%%%%%%%%%%%%%%%%%%%%%%%%%%%%%%%%%%%%%%%%%%%%%%%%%%%%%%
%                         FORMATO DE TESIS FCFM BUAP                              %
%%%%%%%%%%%%%%%%%%%%%%%%%%%%%%%%%%%%%%%%%%%%%%%%%%%%%%%%%%%%%%%%%%%%%%%%%%%%%%%%

% ADAPTADO PARA FCFM-BUAP: @augusto  

\documentclass[twoside,11pt]{Latex/Classes/thesisFCFMBUAP}
%         PUEDEN INCLUIR EN ESTE ESPACIO LOS PAQUETES EXTRA, O BIEN, EN EL ARCHIVO "PhDthesisPSnPDF.cls" EN "./Latex/Classes/"
\usepackage{blindtext}                        % Para insertar texto dummy, de ejemplo, pues.
\usepackage[round, sort, numbers]{natbib}  % Personalizar la bibliografía a gusto de cada quien
% Note:
\include{Latex/Macros/MacroFile1}           % Archivo con funciones útiles


%%%%%%%%%%%%%%%%%%%%%%%%%%%%%%%%%%%%%%%%%%%%%%%%%%%%%%%%%%%%%%%%%%%%%%%%%%%%%%%%
%                                   DATOS                                      %
%%%%%%%%%%%%%%%%%%%%%%%%%%%%%%%%%%%%%%%%%%%%%%%%%%%%%%%%%%%%%%%%%%%%%%%%%%%%%%%%
\title{Titulo de la tesis}
\author{Nombre del reprobado} 
\facultad{Facultad de Ciencias Físico Matemáticas }                 % Nombre de la facultad/escuela
\escudofacultad{Latex/Classes/Escudos/fmed_grande} % Aquí ponen la ruta y nombre del escudo de su facultad, actualmente, la carpeta Latex/Classes/Escudos

\degree{Carrera del alumno}       % Carrera
\director{director de tesis}                   % Director de tesis
%\tutor{Nombre  Tutor }                    % Tutor de tesis, si aplica
\degreedate{año de titulacion y o fecha}                                     % Año de la fecha del examen
\lugar{Lugar de la ejecucion (Puebla, pue)}                        % Lugar

%\portadafalse                              % Portada en NEGRO, descomentar y comentar la línea siguiente si se quiere utilizar
\portadatrue                                % Portada en COLOR



%% Opciones del posgrado (descomentar si las necesitan)
	%\posgradotrue                                                    
	%\programa{programa de maestría y doctorado en fisica}
	%\campo{Fisica}
	%% En caso de que haya comité tutor
	%\comitetrue
	%\ctutoruno{Dr. Emmet L. Brown}
	%\ctutordos{Dr. El Doctor}
%% Datos del jurado                             
	%\presidente{Dr. 1}
	%\secretario{Dr. 2}
	%\vocal{Dr. 3}
	%\supuno{Dr. 4}
	%\supdos{Dr. 5}
	%\institucion{BUAP FCFM}

\keywords{tesis,autor,tutor,etc}            % Palablas clave para los metadatos del PDF
\subject{tema_1,tema_2}                     % Tema para metadatos del PDF  

%%%%%%%%%%%%%%%%%%%%%%%%%%%%%%%%%%%%%%%%%%%%%%%%%%%%%
%                   PORTADA                         %
%%%%%%%%%%%%%%%%%%%%%%%%%%%%%%%%%%%%%%%%%%%%%%%%%%%%%
\begin{document}

\maketitle									% Se redefinió este comando en el archivo de la clase para generar automáticamente la portada a partir de los datos

%%%%%%%%%%%%%%%%%%%%%%%%%%%%%%%%%%%%%%%%%%%%%%%%%%%%%
%                  PRÓLOGO                          %
%%%%%%%%%%%%%%%%%%%%%%%%%%%%%%%%%%%%%%%%%%%%%%%%%%%%%
\frontmatter
\include{Agradecimientos/Dedicatoria}       % Comentar línea si no se usa
\include{Agradecimientos/Agradecimientos}   % Comentar línea si no se usa 
\include{Declaracion/Declaracion}           % Comentar línea si no se usa
\include{Resumen/Resumen}                   % Comentar línea si no se usa

%%%%%%%%%%%%%%%%%%%%%%%%%%%%%%%%%%%%%%%%%%%%%%%%%%%%%
%                   ÍNDICES                         %
%%%%%%%%%%%%%%%%%%%%%%%%%%%%%%%%%%%%%%%%%%%%%%%%%%%%%
%Esta sección genera el índice
\setcounter{secnumdepth}{3} % organisational level that receives a numbers
\setcounter{tocdepth}{3}    % print table of contents for level 3
\tableofcontents            % Genera el índice 
%: ----------------------- list of figures/tables ------------------------
\listoffigures              % Genera el ínidce de figuras, comentar línea si no se usa
\listoftables               % Genera índice de tablas, comentar línea si no se usa


%%%%%%%%%%%%%%%%%%%%%%%%%%%%%%%%%%%%%%%%%%%%%%%%%%%%%
%                   CONTENIDO                       %
%%%%%%%%%%%%%%%%%%%%%%%%%%%%%%%%%%%%%%%%%%%%%%%%%%%%%

\mainmatter
\def\baselinestretch{1.5}                   % Interlineado de 1.5

\chapter{Introducción}



\section{Presentación} 
\section{Objetivo}

\section{Motivación}

\section{Planteamiento del problema}

\section{Metodología}

\section{Contribuciones}


\section{Estructura de la tesis}
            

%%%%%%%%%%%%%%%%%%%%%%%%%%%%%%%%%%%%%%%%%%%%%%%%%%%%%%%%%%%%%%%%%%%%%%%%%
%           Capítulo 2: MARCO TEÓRICO - REVISIÓN DE LITERATURA
%%%%%%%%%%%%%%%%%%%%%%%%%%%%%%%%%%%%%%%%%%%%%%%%%%%%%%%%%%%%%%%%%%%%%%%%%

\chapter{Antecedentes}




           



\chapter{Diseño del experimento}
      
\chapter{Análisis de Resultados}
   
\include{Capitulo5/Capitulo5}            

%%%%%%%%%%%%%%%%%%%%%%%%%%%%%%%%%%%%%%%%%%%%%%%%%%%%%
%                   APÉNDICES                       %
%%%%%%%%%%%%%%%%%%%%%%%%%%%%%%%%%%%%%%%%%%%%%%%%%%%%%
\appendix
\include{Apendices/Apendice1}               

%%%%%%%%%%%%%%%%%%%%%%%%%%%%%%%%%%%%%%%%%%%%%%%%%%%%%
%                   REFERENCIAS                     %
%%%%%%%%%%%%%%%%%%%%%%%%%%%%%%%%%%%%%%%%%%%%%%%%%%%%%
% existen varios estilos de bilbiografía, pueden cambiarlos a placer
\bibliographystyle{apalike} % otros estilos pueden ser abbrv, acm, alpha, apalike, ieeetr, plain, siam, unsrt

%El formato trae otros estilos, o pueden agregar uno que les guste:
%\bibliographystyle{Latex/Classes/PhDbiblio-case} % title forced lower case
%\bibliographystyle{Latex/Classes/PhDbiblio-bold} % title as in bibtex but bold
%\bibliographystyle{Latex/Classes/PhDbiblio-url} % bold + www link if provided
%\bibliographystyle{Latex/Classes/jmb} % calls style file jmb.bst

\bibliography{Bibliografia/referencias}             % Archivo .bib


\end{document}
